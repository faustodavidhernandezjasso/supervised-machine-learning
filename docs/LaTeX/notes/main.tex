\documentclass[a4paper]{article} 
\addtolength{\hoffset}{-2.25cm}
\addtolength{\textwidth}{4.5cm}
\addtolength{\voffset}{-3.25cm}
\addtolength{\textheight}{5cm}
\setlength{\parskip}{0pt}
\setlength{\parindent}{0in}

%----------------------------------------------------------------------------------------
%	PACKAGES AND OTHER DOCUMENT CONFIGURATIONS
%----------------------------------------------------------------------------------------

\usepackage{blindtext} % Package to generate dummy text
\usepackage{charter} % Use the Charter font
\usepackage[utf8]{inputenc} % Use UTF-8 encoding
\usepackage{microtype} % Slightly tweak font spacing for aesthetics
\usepackage[english, ngerman]{babel} % Language hyphenation and typographical rules
\usepackage{amsthm, amsmath, amssymb} % Mathematical typesetting
\usepackage{float} % Improved interface for floating objects
\usepackage[final, colorlinks = true, 
            linkcolor = black, 
            citecolor = black]{hyperref} % For hyperlinks in the PDF
\usepackage{graphicx, multicol} % Enhanced support for graphics
\usepackage{xcolor} % Driver-independent color extensions
\usepackage{marvosym, wasysym} % More symbols
\usepackage{rotating} % Rotation tools
\usepackage{censor} % Facilities for controlling restricted text
\usepackage{listings, style/lstlisting} % Environment for non-formatted code, !uses style file!
\usepackage{pseudocode} % Environment for specifying algorithms in a natural way
\usepackage{style/avm} % Environment for f-structures, !uses style file!
\usepackage{booktabs} % Enhances quality of tables
\usepackage{tikz-qtree} % Easy tree drawing tool
\tikzset{every tree node/.style={align=center,anchor=north},
         level distance=2cm} % Configuration for q-trees
\usepackage{style/btree} % Configuration for b-trees and b+-trees, !uses style file!
\usepackage[backend=biber,style=numeric,
            sorting=nyt]{biblatex} % Complete reimplementation of bibliographic facilities
\addbibresource{ecl.bib}
\usepackage{csquotes} % Context sensitive quotation facilities
\usepackage[yyyymmdd]{datetime} % Uses YEAR-MONTH-DAY format for dates
\renewcommand{\dateseparator}{-} % Sets dateseparator to '-'
\usepackage{fancyhdr} % Headers and footers
\pagestyle{fancy} % All pages have headers and footers
\fancyhead{}\renewcommand{\headrulewidth}{0pt} % Blank out the default header
\fancyfoot[L]{} % Custom footer text
\fancyfoot[C]{} % Custom footer text
\fancyfoot[R]{\thepage} % Custom footer text
\newcommand{\note}[1]{\marginpar{\scriptsize \textcolor{red}{#1}}} % Enables comments in red on margin

%----------------------------------------------------------------------------------------

\newcommand{\pow}[2]{#1^{#2}}
\newcommand{\supra}[1]{\textsuperscript{#1}}
\begin{document}

%-------------------------------
%	TITLE SECTION
%-------------------------------

\fancyhead[C]{}
\hrule \medskip % Upper rule
\begin{minipage}{0.295\textwidth} 
\raggedright
\footnotesize
Fausto David Hernández Jasso \hfill\\   
317000928 \hfill\\
fausto.david.hernandez.jasso@ciencias.unam.mx
\end{minipage}
\begin{minipage}{0.4\textwidth} 
\centering 
\large 
Notes\\ 
\normalsize 
Supervised Machine Learning: Regression and Classification\\ 
\end{minipage}
\begin{minipage}{0.295\textwidth} 
\raggedleft
\today\hfill\\
\end{minipage}
\medskip\hrule 
\bigskip
\section{Introduction}
\subsection{What is machine learning?}
\noindent
According to \textbf{Arthur Samuel} is: "Field of study that gives computers
the ability to learn without being explicitly programmed."
\subsection{Classification of machine learning algorithms}
\noindent
\begin{itemize}
    \item Supervised learning.
    \item Unsupervised learning.
    \item Recommender systems.
    \item Reinforcement learning.
\end{itemize}
\subsection{Supervised Learning}
\noindent
It refers to algorithms that learn \(x\) to \(y\) or input to output mappings.
The key characteristic of supervised learning is that you give your learning 
algorithm examples to learn from that includes the \textit{right answers}.
This it means the correct label \(y\) for a given input \(x\).
\subsubsection{Housing price prediction}
\noindent
We want to predict housing prices based on the size of the house.
\subsubsection{Regression}
\noindent
In this type of supervised learning we are trying to predict a number from infinitely 
many possible numbers.
\subsubsection{Classification}
\noindent
This kind of algorithm predicts categories. Categories don't have to be numbers.
\subsection{Unsupervised Learning}
\noindent
In these kind of algorithms were given data that isn't associated with any output
lables \(y\). Our goal is finding something interesting in unlabeled data. We are not trying to
supervised the algorithm to give some quote right answer for every input, instead we asked the 
algorithm to figure out all by itself. 
\subsubsection{Formal definition}
\noindent
Data only comes with inputs \(x\), but not output labels \(y\). Algorithm has to find a structure.
\subsubsection{Clustering algorithms}
\noindent
It takes data without label and tries to automatically group them into clusters.
\subsubsection{Anomaly detection}
\noindent
Find unusual effects.
\subsubsection{Dimensionaly reduction}
\noindent
This algorithm let you take a big data set and compress it to a much smaller data set lossing 
little information as possible.
\end{document}
